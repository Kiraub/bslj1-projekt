\documentclass{article}

\usepackage[utf8]{inputenc}
\usepackage[ngerman]{babel}
\usepackage[T1]{fontenc}

\title{Projektarbeit 1. Lehrjahr}
\author{Henning Meyer\\Erik Bauer}

\begin{document}

%TODO Titelseite

\maketitle
\newpage
\tableofcontents
\newpage

% Anfang der Arbeit

\section{Analyse und Vorüberlegungen}

\subsection{IST-Analyse}

\subsubsection{Auftraggebersituation}

	Der Auftraggeber für dieses Projekt ist die technische Berufsschule Rostock. Die Schule besitzt mehrere Laborräume, in denen Computer mit gängiger Peripherie wie Maus und Tastatur vorhanden sind. Diese werden in gegebenen Unterrichststunden den Schülern zum Arbeiten zur Verfügung gestellt.
	Auf den Computern befinden sich u.a. Entwicklungsumgebungen für verschiedene Programmiersprachen, darunter die für dieses Projekt relevante Software "Microsoft Visual Studio" in verschiedenen Versionen. Die Rechner verfügen ebenfalls über einen Internetzugang für Recherchezwecke, Anbindung an das lokale Schulnetzwerk mit Speicherplatz für jeden Schüler, sowie Schutzsoftware, um die Funktionalität der lokalen PCs zu gewährleisten.
	
\subsubsection{Auftragnehmersituation}

	Der Auftragnehmer ist in Form eines Projektteams bestehend aus zwei Schülern der Klasse FIN81 gegeben. Beide absolvieren das erste Lehrjahr der Ausbildung zum Fachinformatiker für Anwendungsentwicklung. Es bestehen Vorkenntnisse in unterschiedlichen Programmiersprachen. 
	\textbf{VORKENNTNISSE EINFÜGEN}%TODO Vorkenntnisse
	
	Zum Arbeiten besitzen die Teammitglieder  eigene Computer, es stehen ihnen aber ggf. auch Rechner in ihren Arbeitsplätzen und in der Berufsschule zur Verfügung.

\subsection{SOLL-Analyse}

	Als Anforderung an das Projektteam wurden Kriterien für ein Programm, sowie eine dazugehörige schriftliche Ausarbeitung gegeben. Die gewünschte Anwendung ist hierbei durch Muss-/ und Wunsch-Funktionalitäten beschrieben, konkrete Einschränkungen an die Umsetzung sind dabei aber nicht gemacht worden. Die Hauptaufgabe des Programms liegt in der Visualisierung von physikalischen Messdaten und deren Zusammenhänge in Form von dynamisch erstellten Graphen. Ausserdem muss es zu Präsentationszwecken auf den Schulrechnern lauffähig sein.

\subsection{Problemstellung}

	Es ist eine abstrakte, allgemeine Beschreibung einer elektrotechnischen Schaltung gegeben, aus der im Anwendungsfall relevante physikalische Größen ermittelt werden. Die Aufzeichnung der Werte soll innerhalb des Programms erfolgen und in einem Koordinatensystem veranschaulicht werden. Die Zusammenhänge zwischen den messbaren bzw. berechenbaren Werten, sind vom Programm automatisch als Graph darzustellen. Speziell soll der Graph in einem begrenzten Maße ohne Nutzung fremden Quellcodes gezeichnet werden. Des Weiteren sind die Graphen nur als Näherungskurve an die experimentell ermittelten Messdaten darzustellen.
	%TODO Physikalische Analyse

\newpage
\section{Lösungsfindung und Umsetzung}

\subsection{Betrachtung von Lösungsvarianten}

	Nach Analyse der gegebenen Anforderungen entschied das Projektteam, dass eine Entwicklungsumgebung, welche das Erstellen graphischer Oberflächen unterstützt, angemessen zur Lösung ist. Ausserdem sollte die verwendete Programmiersprache eine einfache Schnittstelle oder Möglichkeit bieten, vom Programmierer definierte Grafiken auf dem Bildschirm anzuzeigen. Hierbei sollten speziell einzelne Punkte oder gerade Linien darstellbar sein, damit die Zeichnung von Kurven gut implementierbar ist.
	Als mögliche Programmiersprachen wurden \textit{\glqq Python\grqq{}}, unter Nutzung von \textit{\glqq TKinter\grqq{}}, \textit{\glqq Java\grqq{}}  und \textit{\glqq C\#\grqq{}} betrachtet.
	Zusätzlich zur Seite der Problemstellung, wurde dann die Seite der Vorkenntnisse in Betracht gezogen. Mit den gefundenen Kriterien und den Überschneidungen vorhandener Entwicklungserfahrungen, hat sich das Team letztlich für die Nutzung der Sprache \textit{\glqq C\#\grqq{}}, unterstützt durch die Entwicklungsumgebung \textit{\glqq Microsoft Visual Studio\grqq{}}, entschieden.

\subsection{Umsetzung der gewählten Variante}

	\textbf{PLACEHOLDER}%TODO Umsetzung

\newpage
\section{Testen der Anwendung}

\subsection{Betrachtete Testfälle}

	%TODO Testfälle

\subsection{Testprotokolle}

	%TODO Testprotokolle

\newpage
\section{Quellen}

	%TODO Quellen

% Ende der Arbeit

\end{document}